%%%%%%%%%%%%%%%%%%%%%%%%%%%%%%%%%%%%%%%%%%%%%%%%%%%%%%%%%%%%%%%
%%                   RobSIC Beamer Example                   %%
%%                                                           %%
%% This is an example presentation showing how to use the    %%
%% RobSIC custom template. To create your own, just copy     %%
%% this file to your project and add your own slides,        %%
%% using the ones here for guidance. When finished, just     %%
%% delete the example slides and you're done.                %%
%%                                                           %%
%% Authors: Wendell F. S. Diniz (wendelldiniz@unifei.edu.br) %%
%%                                                           %%
%% Created: September, 2022                                  %%
%% Licensing: MIT License                                    %%
%%                                                           %%
%%%%%%%%%%%%%%%%%%%%%%%%%%%%%%%%%%%%%%%%%%%%%%%%%%%%%%%%%%%%%%%

%% THE BEAMER DOCUMENT CLASS %%
%
% By default, we use modern screen size with 16:9 aspect ratio.
% You can change the option if your presentation media requires it.
\documentclass[12pt, xcolor=table, aspectratio=169]{beamer}

% The RobSIC beamer theme.
% Available options
% option=full (full theme, with section titles)
% option=simple (simplified theme)
\usetheme[option=full]{robsic}

%% PACKAGES %%
\usepackage[utf8]{inputenc}  % utf8 support
\usepackage[T1]{fontenc}     % Font encoding
\usepackage[portuges]{babel} % Language control
\usepackage{hyperref}        % Hyperlink manager
\usepackage{nameref}         % Reference to objects
\usepackage{multicol}       % Break content in columns

%% PACKAGE CONFIGS %%

\hypersetup{colorlinks,linkcolor=,urlcolor=accent} % URL colors

%% CUSTOM COMMANDS %%

%% Current section name on frame titles
\makeatletter
\newcommand*{\currentname}{\@currentlabelname}
\makeatother

%% PREAMBLE %%

\title{Exemplo de Apresentação utilizando o Modelo Oficial do RobSIC}
\author[Autor]{Nome Completo}
\institute[RobSIC]{RobSIC - Laboratório de Robótica, Sistemas Inteligentes e Complexos}
\date[SIGLA 22]{Título da conferência - \today} % Change it to your convenience

%% THE DOCUMENT %%

\begin{document}

% The title frame
\begin{frame}
  \titlepage
\end{frame}

% Using tocdepth to show only sections in the summary.
% Recommended when you have too many subsections or too
% many sections.
% \setcounter{tocdepth}{1}

% Comment the following section and frame, if you
% do not want a Summary frame

\begin{frame}
  \frametitle{\currentname}
% If your summary is small enough, use the one column approach

%   \tableofcontents
  
% Use the below set of commands to automatically split
% a long summary in two columns. Break on a list of sub-
% sections may occur.

%   \begin{multicols}{2}
%     \tableofcontents
%   \end{multicols}

% Use the below set of commands to manually control the
% breaking point of long summaries.

  \begin{columns}[onlytextwidth,T]
    \begin{column}{.45\textwidth}
      \tableofcontents[sections=1-2]
    \end{column}
    \begin{column}{.45\textwidth}
      \tableofcontents[sections=3-5]
    \end{column}
  \end{columns}
  
\end{frame}



% Use sections and subsections to organize your talk
% Rarely, use subsubsections, but it is not recommended
% to go beyond subsections in presentations.
\section[Introdução]{Introdução}

\subsection[Apresentação]{Apresentação}

\begin{frame}
 \frametitle{\currentname}
 \begin{itemize}
  \item Bem vindo ao Modelo Oficial de Apresentação do RobSIC!
  \item O modelo é construído sobre a classe Beamer utilizando o sistema de tipografia \LaTeX.
  \item Este documento serve como \textit{template} para a criação de suas próprias apresentações e também como documentação do modelo.
  \item O modelo pode ser obtido através de seu \href{https://github.com/Robsic/beamer-template-robsic}{repositório oficial}.
 \end{itemize}

\end{frame}

\subsection{Como usar este modelo}

\begin{frame}
 \frametitle{\currentname}
 \begin{itemize}
  \item Para utilizar este modelo, basta clonar o repositório e reescrever o arquivo \texttt{robsic-example.tex}.
  \item Se preferir começar um novo arquivo, acrescente o comando \texttt{{\textbackslash}usetheme[<opções>]\{robsic\}}.
 \end{itemize}

\end{frame}

\subsection{Variações do tema}

\begin{frame}
 \frametitle{\currentname}
 \begin{itemize}
  \item O modelo robsic oferece duas variações de tema que podem ser configuradas no comando \texttt{{\textbackslash}usetheme}:
  \end{itemize}
 \begin{description}
  \item[option] Escolhe uma das duas variações do tema, \alert{simple} ou \alert{full}.
 \end{description}
 
 \begin{exampleblock}{Exemplo}
  \begin{itemize}
   \item \alert{Tema simples:}\linebreak
         \texttt{{\textbackslash}usetheme[option=simple]\{robsic\}}
   \item \alert{Tema completo:}\linebreak
         \texttt{{\textbackslash}usetheme[option=full]\{robsic\}}
  \end{itemize}
 \end{exampleblock}
\end{frame}

\begin{frame}
 \frametitle{\currentname}
  \begin{alertblock}{Cuidado!}
  Não passar uma opção desativará todos os formatos do tema.
 \end{alertblock}
\end{frame}

\section{Construindo os quadros}

\subsection{Título e subtítulo do quadro}

\begin{frame}
  \frametitle{\currentname}
  \begin{itemize}
    \item O \alert{título} e o \alert{subtítulo} do quadro podem ser definidos usando-se os comandos
          \texttt{{\textbackslash}frametitle\{\}} e \texttt{{\textbackslash}framesubtitle\{\}}.
    \item O conteúdo é exibido na área reservada em ambas as variações do modelo.
    \item No modelo, está definido o comando \texttt{{\textbackslash}currentname}, que imprime o nome da seção ou subseção corrente.
    \item É boa prática colocar pelo menos um título em cada quadro. O subtítulo é opcional.
    \item Quadros com figuras, entretanto, podem dispensar o título. O modelo está preparado para não exibir a área reservada, se nenhum título for definido.
  \end{itemize}
\end{frame}

\subsection{Exibindo o conteúdo}
\begin{frame}
\frametitle{\currentname}
% Use itemize or enumerate to show the points of your talk.
% Use short, but significative sentences. Your talk is the
% main attraction of the presentation. The frames should be
% a support to your talk.
  \begin{itemize}
    \item Utilize o ambiente \texttt{{\textbackslash}itemize} para exibir seus pontos.
    \item Use sentenças sucintas, mas que transmitam as ideias claramente.
    \item Se um item ficar muito grande, não exite em dividi-lo em itens menores.
    \item Lembre-se, sua fala é a estrela do show. Os \textit{slides} devem ser um apoio visual a sua fala.
  \end{itemize}
  \begin{enumerate}
   \item Listas numeradas são possíveis com o ambiente \texttt{{\textbackslash}enumerate}.
   \item Utilize-as em conjunto com \texttt{{\textbackslash}itemize}.
  \end{enumerate}

\end{frame}

\subsection{Usando camadas (overlays)}
\begin{frame}
\frametitle{\currentname}

% Use itemize or enumerate to show the points of your talk.
% Use short, but significative sentences. Your talk is the
% main attraction of the presentation. The frames should be
% a support to your talk.
\begin{itemize}
  \item Ovelays são uma maneira fácil de revelar o conteúdo por partes.
  \item<1-|alert@1> Por exemplo, este item estará destacado. \only<2->{Mas apenas na primeira camada.}
  \item<2-> Os outros itens são revelados passo a passo.
  \item<3-> Repare que o número da página é o mesmo.
  \item<4-> Todos os itens pertecem a um único quadro, dividido em camadas (overlays).
  \item<5-> Mais detalhes podem ser encontrados no \href{https://tug.ctan.org/macros/latex/contrib/beamer/doc/beameruserguide.pdf}{manual da classe Beamer}
\end{itemize}
\end{frame}

\subsection[Usando blocos]{Usando blocos}

\begin{frame}
 \frametitle{\currentname}
 \begin{itemize}
  \item O ambiente de bloco cria uma caixa decorada com título e conteúdo, para destacar pontos de interesse especial, equações, teoremas, etc.
  \item Existem três ambientes distintos, com formatação também distinta entre si.
  \item \texttt{{\textbackslash}begin\{block\}\{título\}...{\textbackslash}end\{block\}} insere um bloco de uso geral.
  \item \texttt{{\textbackslash}begin\{exampleblock\}\{título\}...{\textbackslash}end\{exampleblock\}} cria um bloco de cor diferente, usado para apresentar exemplos e ponderações.
  \item \texttt{{\textbackslash}begin\{alertblock\}\{título\}...{\textbackslash}end\{alertblock\}} insere um bloco de alerta, com cor destacada. Use-o para pontos de especial importância.
  \end{itemize}
\end{frame}

\begin{frame}
  \frametitle{\currentname}
  \begin{block}{Bloco normal}
   Este é um bloco geral, usado para destacar pontos principais.
  \end{block}
  \begin{exampleblock}{Bloco de Exemplo}
    Este é um bloco de exemplo, com cor diferente, para destacar exemplos e ponderações.
  \end{exampleblock}
  \begin{alertblock}{Bloco de Alerta}
    Este é um bloco de alerta, que deve ser usado para destacar pontos de importância elevada.
  \end{alertblock}
\end{frame}

\begin{frame}
  \frametitle{\currentname}
  \begin{block}{Overlays e blocos}<1-2|alert@1>
   Os comandos de overlay também podem ser utilizados em blocos! \only<2->{Neste caso, temos um 
   destaque na primeira camada e um texto exibido apena na segunda camada.}
  \end{block}

\end{frame}

\subsection{Inserindo Equações}

\begin{frame}
 \frametitle{Inserindo Equações}
 \begin{block}{Equação de Pitágoras}
   \begin{equation*}
     a^2 = b^2 + c^2
   \end{equation*}
 \end{block}
\end{frame}


\subsection{Inserindo Figuras}

\begin{frame}
  \frametitle{\currentname}
\end{frame}


\subsection{Inserindo bibliografia}
\begin{frame}
  \frametitle{\currentname}
\end{frame}

\end{document}
