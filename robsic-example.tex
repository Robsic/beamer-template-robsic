%%%%%%%%%%%%%%%%%%%%%%%%%%%%%%%%%%%%%%%%%%%%%%%%%%%%%%%%%%%%%%%
%%                   RobSIC Beamer Example                   %%
%%                                                           %%
%% This is an example presentation showing how to use the    %%
%% RobSIC custom template. To create your own, just copy     %%
%% this file to your project and add your own slides,        %%
%% using the ones here for guidance. When finished, just     %%
%% delete the example slides and you're done.                %%
%%                                                           %%
%% Authors: Wendell F. S. Diniz (wendelldiniz@unifei.edu.br) %%
%%                                                           %%
%% Created: September, 2022                                  %%
%% Licensing: MIT License                                    %%
%%                                                           %%
%%%%%%%%%%%%%%%%%%%%%%%%%%%%%%%%%%%%%%%%%%%%%%%%%%%%%%%%%%%%%%%

% The beamer document class
%
% By default, we use modern screen size with 16:9 aspect ratio.
% You can change the option if your presentation media requires it.
\documentclass[12pt, xcolor=table, aspectratio=169]{beamer}

% The RobSIC beamer theme.
% Available options
% navigation=false (disables the navigation bar)
% option=full (full theme, with section titles)
% option=simple (simplified theme)
\usetheme[navigation=false, option=full]{robsic}

% Font configuration packages
\usepackage[utf8]{inputenc}
\usepackage[T1]{fontenc}

% Optional packages

%% PREAMBLE %%

\title{Modelo de Apresentação do RobSIC}
\author[Author]{Nome Completo}
\institute{RobSIC - Robótica, Sistemas Inteligentes e Complexos}
\date[SIGLA 22]{Título da conferência}
%% CUSTOM COMMANDS %%

%% THE DOCUMENT %%

\begin{document}

\begin{frame}
  \titlepage
\end{frame}

\mode<beamer>{
\begin{frame}
  \frametitle{Outline}
  \tableofcontents
\end{frame}
}

\section[Intro]{Introduction}
\begin{frame}
\frametitle{There Is No Largest Prime Number}
\framesubtitle{The proof uses \textit{reductio ad absurdum}.}
\begin{theorem}
There is no largest prime number.
\end{theorem}
\begin{enumerate}
\item<1-| alert@1> Suppose $p$ were the largest prime number.
\item<2-> Let $q$ be the product of the first $p$ numbers.
\item<3-> Then $q+1$ is not divisible by any of them.
\item<1-> But $q + 1$ is greater than $1$, thus divisible by some prime
number not in the first $p$ numbers.
\end{enumerate}
\end{frame}

\section[Math]{Inserindo Equações}

\begin{frame}
 \frametitle{Inserindo Equações}
 \begin{block}{Equação de Pitágoras}
   \begin{equation*}
     a^2 = b^2 + c^2
   \end{equation*}
 \end{block}
\end{frame}


\end{document}
